%% report_template.tex
%% V1.0
%% 2012-03-16
%% by Jesper Pedersen Notander
%% See:
%% http://www.cs.lth.se/jesper_pedersen_notander
%% for current contact information.
%% V1.1
%% 2014-02-20
%% by Per Runeson
%% See:
%% http://www.cs.lth.se/per_runeson
%% for current contact information.
%%
%% This is a template file contaning instructions and a skeleton outline 
%% for the final report in the course ETSA05: Software Engineering 
%% Process - Soft Issues, given by the Department of Computer Science at 
%% Lund University, Sweden.
%% 
%% This template requires IEEEtran.cls, written by Michael Shell, version 
%% 1.7 or later.
%%
%% Support sites:
%% http://www.cs.lth.se/etsa05/
%% http://www.ieee.org/

%%*************************************************************************
%% Legal Notice:
%% This code is offered as-is without any warranty either expressed or
%% implied; without even the implied warranty of MERCHANTABILITY or
%% FITNESS FOR A PARTICULAR PURPOSE! 
%%
%% User assumes all risk.
%%
%% In no event shall Lund University or any contributor to this code be 
%% liable for any damages or losses, including, but not limited to, 
%% incidental, consequential, or any other damages, resulting from the use 
%% or misuse of any information contained here.
%%
%% All comments are the opinions of their respective authors and are not
%% necessarily endorsed by Lund University.
%%
%% This work is distributed under the LaTeX Project Public License (LPPL)
%% ( http://www.latex-project.org/ ) version 1.3, and may be freely used,
%% distributed and modified. A copy of the LPPL, version 1.3, is included
%% in the base LaTeX documentation of all distributions of LaTeX released
%% 2003/12/01 or later.
%%
%% Retain all contribution notices and credits.
%% ** Modified files should be clearly indicated as such, including  **
%% ** renaming them and changing author support contact information. **
%%
%% File list of work: report_template.tex
%%*************************************************************************


\documentclass[conference]{IEEEtran}	
% If IEEEtran.cls has not been installed into the LaTeX system files,
% manually specify the path to it like:
% \documentclass[conference]{../sty/IEEEtran}	   
\usepackage[utf8]{inputenc}
\usepackage{hyperref}

\hypersetup{
 pdfborder={0 0 0},				% Ingen "ram".
 pdftitle={The Android Open Source Project},	% Dokumenttitel.
 pdfauthor={Alexander Andersson,
	Björn Isacsson,
	Björn Lindquist,
	Timmy Larsson},				% Författaren.
 colorlinks=false, 				% false: boxade länkar; true: färgade länkar
 linkcolor=red,					% Färg på interna länkar
 citecolor=green,				% Färg på länkar till referenser
 filecolor=magenta,				% Färg på fillänkar
 urlcolor=cyan					% Färg på externa länkar
}

\begin{document}


\title{Social Aspects of the Android Open Source Project}

% author names and affiliations
% use a multiple column layout for up to three different
% affiliations
\author{
%\IEEEauthorblockN{Name/s per 1st Affiliation (Author)}
%\IEEEauthorblockA{line 1 (of Affiliation): dept. name of organization\\
%line 2: name of organization, acronyms acceptable\\
%line 3: City, Country\\
%line 4: e-mailaddress if desired}
%\and
\IEEEauthorblockN{Alexander Anderson, Björn Isacsson, Timmy Larsson and Björn Lindquist\\}
\IEEEauthorblockA{Students at Lund University\\0708935807
Lund, Sweden\\
E-mail addresses: \href{mailto:dat12aan@student.lu.se}{Alexander Anderson}, \href{mailto:dat12bis@student.lu.se}{Björn Isacsson},  \href{mailto:bas11tla@student.lu.se}{Timmy Larsson}, \href{mailto:dat12bli@student.lu.se}{Björn Lindquist}}}


\maketitle


\begin{abstract}
%This document is a template. The various components of your paper (title, headings, etc.) are already defined on the style sheet, as illustrated by the portions given in this document. The abstract of the final report should present i) background or context, ii) objectives or aims, iii) method for the study, iv) results, and v) conclusions.

This report was written as part of a course on soft issues in software engineering. The report studies the social aspects of the Android Open Source Project, discussing the ethical, legal and business aspects of the project. It also looks into the quality characteristics and availability for the disabled. This was done by studying what is included in the Android Open Source Project, as well as with the help of peer-reviewed and other cited sources. 

%Results
%Conclusions

\end{abstract}


\section{Introduction}
%The introduction section can be used to introduce the company in general or to introduce the purpose and context of this report. This template is a document that provides the predefined outline of your group essay from the seminar compendium. If you want to change or adjust your outline, you must do so before the outline is due (April 11, 2014). The outline shall include a brief description of your system in Section~\ref{system}, and the results from the seminars in Sections~\ref{quality}, \ref{availability}, and \ref{legal} --- at least an itemized form. Further, literature that you base your analysis on should be listed. Primarily, \emph{peer-reviewed literature} shall be used.

%It is important to note that you have a page limit of 5-7 pages (excluding references) for your final report (which is due May 16, 2014), hence it is important to decide how much should be analyzed, discussed, and written for each section.

The company behind the Android Open Source Project is Google\cite{android}, a company that is well-known for their search engine, google.com. They are also widely known for being very popular among computer enthusiasts, but due to the massive amount of information they gather for personalising ads, many also oppose them. In 2008 they also released their operating system for cell phones, namely Android\cite{android-release}. When doing so, they also released it as open source\cite{android}. This includes the entire operating system, with the exception of Google's own apps, that are not included. By releasing the operating system as open source through the Android Open Source Project, they have allowed developers to produce their own versions of the system completely free, making Google's Android one of the largest competitors in the cell phone operating system market\cite{android-market}.


\section{Description of the System}
\label{system}
%In this section you should describe the system that your group has selected. Your system description should not only describe the technical parts of the system, but also from a user point of view. 

The Android Open Source Project is a project created and led by Google. It's an operating system for cell phones, and the entire code is open source, i.e. available for anybody to see and use. It also allows companies and end-users to create their own versions of the operating system and even contribute those to make them openly available to anyone else who might be interested\cite{android}. The Android Open Source Project was started along with the release of Android in 2008. Since then, many large cell phone companies, most famously Samsung, Sony and HTC, have taken on android as the standard operating system for their cell phones. 

The open source code also allows developers to easily write apps for Android. With everything being openly available, it is clear what is and isn't possible to create. It also becomes a lot easier to connect apps to hardware such as the camera or gyroscope.

From a users perspective, Android is often the preferred choice for tech geeks, due to the ability to freely modify it through the open source code. However, compared to Apple's iPhone, the extra options adds many layers of more complexity to Android which shuns those who prefer a simpler \footnote{simplicity in this context refers to the ISO/IEC 9126\cite{jung2004} subcharacteristics ''intuitiveness'' and ''minimalism'' whereas an experienced Linux geek would want to define it as the Arch Linux community does\cite{archway}.} operating system. 


\section{Quality Characteristics}
\label{quality}
%Under this section you should discuss which quality characteristics are particular important for this system. Which quality characteristics are considered most important, i.e. prioritize among the relevant quality characteristics, and provide a rationale for the prioritization. In addition, from which perspective is the prioritization important. You find the quality characteristics in the ISO 9126 standard in one article, e.g. \cite{jung2004} in the articles course compendium. The prioritization can be performed at the characteristics level and/or at the sub-characteristics level.

Using ISO/IEC 9126\cite{jung2004} to investigate the system we found that: \\
For a system like AOSP there is a great need for usability because if people can't use the phone all other quality aspects become irrelevant. The second most important aspect of the AOSP is its portability. Its ability to be run on different hardware is part of its appeal to electronics producers and by extension the end consumers. Thirdly being designed to run on smartphones, tabelts and generally low performance platforms makes the efficiency of AOSP highly important as it would restrict the battery life and usability of devices running AOSP. As integrated as smartphones are in our lives reliability of the software is crucial. If the device doesn't perfom and function as expected its usability will suffer heavily. The continuous use and expansion of AOSP requires easy maintenance as to keep it relevant. Functionality is mainly provided by third party software and the AOSP only requires framework to allow them to function. There are several quality guidelines for how to write third party software for AOSP\cite{android-quality}.


\section{Availability for Disabled}
\label{availability}

%Here you should provide a discussion/analysis of which aspects of availability for the elderly, disabled, and individuals with special needs are current interests. Analyze how can Information and Communication Technology (ICT) help people with disability. However, if your system is not applicable for people with disabilities, analyze the system with respect to accessibility with respect to general human-computer interaction. 

Android open source project (AOSP) is in it's nature as a smartphone and tablet operating system not suited for several typed of disabilities. Touch input with first and foremost visual feedback hinders anyone with impeded visual and or motorskills. AOSP has several features to help people with visual impairment such as a text-to-speech engine and screen reader\cite{android-disabled-help}. The AOSP project in iteslf does not conatain many additional functions for helping disabled however the functionality of the AOSP can be extended with apps that are specialied for disabled \cite{android-disabled-apps}. Many of the apps Google usually includes in their distribution of AOSP include optimisations for text-to-speech\cite{android-disabled-help}. The AOSP also supports apps that help people with other disabilies such as diabetes\cite{android-disabled-diabetes} by allowing them to use their AOSP based phone to remember, calculate and notify them about theri status. The FDA has proposed guidelines for apps used for medical applications\cite{android-disabled-FDA}.



\section{Ethical Aspects}
\label{ethics}
%Which ethical questions must be answered for the system? Here you should identify, and discuss, ethical dilemmas/potential ethical issues related to your system. Examples of different kinds of ethical dilemmas can be found in Berenbach and Broy \cite{berenbach2004}.

Due to the fact that Google themselves do not develope any of the open-source applications and systems they do not themselves control any ethical dilemas that may come up due to the open-source. This could be a problem due to the fact that android automatically is associated with Google. While Google are in charge of the licenses for open-source they have outsourced it to Apache.\cite{android-licenses} This can lead to a major oversight with companies and countries using Android Open-Source Project (AOSP). Countries such as China have used Android as a look-to when writing their own systems \cite{country-license} although their systems i Linux Kernel based. China is known for using technology to control and moniter their citizens. This may become a problem for Android if they are associated with systems such as China's.\\
Applications and System based on AOSP can easily, without Google's knowledge, both break legal laws and ethical rules. So while AOSP is good for the technologic community it can also become a burden for Google and her interests.


\section{Legal Aspects}
\label{legal}
%In this section you should discuss and analyze which legal aspects, e.g. intellectual property, are relevant for your system.

The Android Open Source project is mainly licensed under the Apache Software License 2.0 \cite{apache-license}, something that has been criticised by, among others, Richard Stallman for not being Free Software\cite{rms-android}. Other parts of the system, namely the Linux kernel and applicable patches, are (inherently) licensed with the GPLV2 license\cite{gplv2}\cite{android-licenses}. Despite using Open Source to encourage participatory culture, Google holds a handful of patents that could be used to stifle competition\cite{google-slide-unlock}\cite{google-radial-menu}.


\section{Business Aspects}
\label{financial}
%Under this section business aspects should be analyzed and discussed. What are the business driving forces in favor of and against your system? Who are the investors, who can make a profit, what is the cost for the individual, and what is the cost for society? Which business model is used for your system, and what investment strategy is used (short-term or long-term)? These questions are examples of what can be analyzed with regards to business aspects. 

Google does not make any direct money of Android Open Source Project (AOSP), due to the fact that licensing is free.\cite{Money} The way for Google to make money is to push out android phones, and the fact that they use Google licensed product such as Chrome, Google Search, etc. Google's search engine is today two thirds of the company revenue service and makes up for any loses that are made from AOSP. To be able to use AOSP, one must seek either a individual grant or an company grant. \cite{android-licenses} These licenses are free and Google takes none of the profit generated by products or systems that uses an Appache License.


\section{Summary}
%In the summary section you should summarize what your report is about and present your main findings/consequences. 

In conclusion, The Android Open Source Project has been largely successfull with spreading the operating system, obtaining developers and encouraging participation from the community. Despite Google not seeing any significant revenue streams from the project itself, it acts as a gateway-product to introduce (and lock) customers to their other products, adding to their vast data collections.


\section{Contribution Statement}
%In this section you state the role and contribution of each co-author. 

Co-Author Björn Lindquist has been in charge of researching and writing the Business Aspects and Ethical Aspects of this paper. He has also spent time peer-reviewing the other co-authors' pieces.
\\
\\
Co-Author Alexander Anderson has been in charge of researching and writing the Availability for Disabled part of this paper and has spent time peer-reviewing the other co-authors' pieces.
\\
\\
Co-Author Björn Isacsson has been in charge of writing the Abstract, the Introduction and the Description of the System. Furthermore, he has searched for and found relevant peer-reviewed material for all and any part of the report. He has also spent time peer-reviewing the other co-authors pieces. 
\\
\\
Co-Author Timmy Larsson has been in charge of researching and writing the Legal Aspects and Summary of this paper. Hir has also spent time reviewing the other co-authors' pieces.

% references section
\begin{thebibliography}{1}

\bibitem{jung2004}
H-W. Jung, S-G Kim, and C-S Chung, ''Measuring Software Product Quality: A Survey of ISO/IEC 9126'' IEEE Software, 21(5), pp. 88--92, 2004.

%\bibitem{berenbach2004}
%B. Berenbach, and M. Broy, ''Professional and ethical dilemmas in software engineering'' Computer, 42(1), pp. 74--88, 2004

\bibitem{android}
Android Developers. ''Welcome to the Android Open Source Project!'' Program documentation. N.p., n.d. Web. 30 Mar. 2014. \url{https://source.android.com/}.

\bibitem{android-release}
Dan Morrill ''Announcing the Android 1.0 SDK, release 1'' Initial release of Andropid N.p., n.d. \url{http://android-developers.blogspot.in/2008/09/announcing-android-10-sdk-release-1.html}.

\bibitem{android-market}
Gartner, Inc. "Gartner Says Annual Smartphone Sales Surpassed Sales of Feature Phones for the First Time in 2013." Gartner. Gartner, 13 Feb. 2014. Web. 4 Apr. 2014. \url{http://www.gartner.com/newsroom/id/2665715}.

\bibitem{apache-license}
The Apache Software Foundation ''Apache License, Version 2.0'' January 2004 \url{http://www.apache.org/licenses/LICENSE-2.0}.

\bibitem{rms-android}
Richard Stallman ''Is Android really free software?'' 19 September 2011 The Guardian \url{http://www.theguardian.com/technology/2011/sep/19/android-free-software-stallman}.

\bibitem{gplv2}
GNU GENERAL PUBLIC LICENSE Version 2, June 1991 \url{https://www.kernel.org/pub/linux/kernel/COPYING}.

\bibitem{android-licenses}
Android Open Source Project ''Licenses'' \url{http://source.android.com/source/licenses.html}.

\bibitem{google-radial-menu}
Jack Purcher ''Google Patent Reveals New Radial Menu Design for Android'' Marh 14 2014 \url{http://www.patentbolt.com/2014/03/google-patent-reveals-new-radial-menu-design-for-android.html}

\bibitem{google-slide-unlock}
United States Patent 8,504,842 \url{http://patft.uspto.gov/netacgi/nph-Parser?Sect1=PTO2&Sect2=HITOFF&u=\%2Fnetahtml\%2FPTO\%2Fsearch-adv.htm&r=7&p=1&f=G&l=50&d=PTXT&S1=\%2820130806.PD.+AND+Google.ASNM.\%29&OS=ISD/20130806+AND+AN/Google&RS=\%28ISD/20130806+AND+AN/Google\%29}

\bibitem{Money}
Mark Morelli ''Android no cash cow for Google'' Website Commentary N.P., n.d Web. 11 Dec. 2013 \url{http://www.fool.com/investing/general/2013/12/11/android-is-no-cash-cow-for-google.aspx}.

\bibitem{android-market}
Gartner, Inc. "Gartner Says Annual Smartphone Sales Surpassed Sales of Feature Phones for the First Time in 2013." Gartner. Gartner, 13 Feb. 2014. Web. 4 Apr. 2014. \url{http://www.gartner.com/newsroom/id/2665715}.

\bibitem{android-disabled-help}
Google ''Using Google products: How to use accessibility features'' url{http://www.google.com/accessibility/products/}

\bibitem{android-disabled-apps}
Google ''Din Magnifier" \url{https://play.google.com/store/apps/details?id=com.yuvalluzon.yourmagnifier}

\bibitem{android-disabled-diabetes}
Medivo ''OnTrack Diabetes'' app for monitoring relevantdiabetes information
\url{https://play.google.com/store/apps/details?id=com.gexperts.ontrack]}

\bibitem{android-disabled-FDA}
FDA ''Draft Guidance for Industry and Food and Drug Administration Staff; Mobile Medical Applications; Availability''
\url{http://www.regulations.gov/#!documentDetail;D=FDA-2011-D-0530-0001}

\bibitem{archway}
ArchWiki ''The Arch Way'' \url{https://wiki.archlinux.org/index.php/The\_Arch\_Way#Simplicity}

%\bibitem{android-jail}
%Ron Amadeo ''Google’s iron grip on Android: Controlling open source by any means necessary'' Google policies regarding android N.p., n.d. Oct 21 2013, 3:00am CEST \url{http://arstechnica.com/gadgets/2013/10/googles-iron-grip-on-android-controlling-open-source-by-any-means-necessary}

%\bibitem{android-plattforms}
%Mark LaPedus ''Update: MIPS gets sweet with Honeycomb'' Android running on MIPS N.p., n.d. 4/26/2011 06:37 PM EDT \url{http://www.eetimes.com/document.asp?doc_id=1259370}

%\bibitem{android-license}
%Android Developers. ''Licenses'' Licenses for Android N.p, n.d. Viewed 2014-4-4 \url{http://source.android.com/source/licenses.html}

%\bibitem{android-powered-by-android-mandate}
%Russel Holly ''Google mandates ‘Powered by Android’ branding on new devices'' Google mandates that devices running Android must show the powered by anderoid at startup N.p, n.d. \url{http://www.geek.com/android/google-mandates-powered-by-android-branding-on-new-devices-1589253/}

\bibitem{country-license}
Jim Lynch ''Did China Copy Android''  20 Jan 2014 \url{http://www.itworld.com/open-source/400961/did-china-copy-android-its-new-mobile-operating-system} 

\bibitem{android-quality}
Android developer ''Core App Quality Guidelines''
\url{http://developer.android.com/distribute/googleplay/quality/core.html}

\end{thebibliography}

\end{document}


